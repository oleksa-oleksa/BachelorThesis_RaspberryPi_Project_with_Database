\appendix
\chapter{Anlagen}

\section{mikroBUS}
\label{sec:appendix:microbus}
Der Standard legt das physikalische Layout der mikroBus-Pinout-Verbindung, die verwendeten Kommunikations- und Stromversorgungspins auf dem Mainboard fest. Der Zweck von mikroBUS ist es, eine einfache Erweiterbarkeit der Hardware mit einer großen Anzahl von standardisierten kompakten Zusatzboards zu ermöglichen, von denen jede einen einzelnen Sensor, Display, Encoder oder Motortreiber, eine integrierte Schaltung hat. Der von MikroElektronika entwickelte mikroBUS ist ein offener Standard - jeder kann mikroBUS in seinem Hardwaredesign implementieren. Die Abbildung\footnote{https://download.mikroe.com/documents/standards/mikrobus/mikrobus-standard-specification-v200.pdf} \ref{fig:mikrobus} zeigt die Pinout Spezifikation des Herstellers, die man entsprechend ändern kann und die neue Verbindungen für den eigenen Projekt feststellen. Wenn ein Modul eine Schnittstelle verwendet, die bereits auf mikroBUS vorhanden ist, benutzt man diese exakten Pins und markiert diese entsprechend. Wenn ein Pin nicht verwendet wird, sollte er als NC (für "Not Connected") markiert sein. 
\begin{figure}[!htb]
	\centering
	\fbox{\includegraphics[width=1\textwidth]{gfx/microbus.png}}
	\caption{MikroBus Pinout Spezifikation}
	\label{fig:mikrobus}
\end{figure}

\section{Cloud Computing}
\label{sec:appendix:cloud}
Die \textit{Cloud} oder \textit{Cloud Computing} Begriff kommt offensichtlich aus dem Englischen und heißt auf Deutsch "Wolke". Der Begriff beschreibt einen oder mehrere entfernte Server, auf die man seine Daten von einem Gerät über das Internet hochladen kann. Dann übernimmt die Cloud die Aufgaben wie die Datenverarbeitung oder komplizierte Programmabläufe. Während der Datenverarbeitung weißt der Nutzer nicht, wie viele Server hinter der Cloud stecken und welche komplizierte Hardware für die Berechnungen benötigt werden.\\\\
Der Name \textit{Cloud} stammt davon, dass Cloud Computing quasi wolkig (unklar) für den Nutzer ist, weil ein Nutzer nicht wissen muss und kann, wo sich der Server physikalisch befindet. Selbst wenn ein Server ausfällt, hat dies keine Auswirkungen auf das gesamte System und diese Eigenschaft nennt man "Wolke". Cloud hilft bei IoT-Anwendungen, da die IoT-Hersteller die anfallenden Datenmengen mit der eigenen IT-Infrastruktur in der Regel nicht bewältigen und umfassend nutzen können. Cloud bietet die nötige Flexibilität, schnelle Skalierbarkeit und ständige Verfügbarkeit ohne langfristige, kapitalintensive Investitionen in eigene Rechenzentren.\\\\
Cloud erlaubt auch mehreren Nutzer einen gemeinsamen Zugriff auf die auf dem Server gespeicherten Dateien zu bekommen. So kann man mit einem berühmten Cloud Service von Google namens \textit{"Google Drive"\footnote{https://drive.google.com/drive/my-drive}} an einer Präsentation oder anderen Dokumenten zusammenarbeiten. Man kann die von den jeweiligen Benutzern am Dokument vorgenommenen Änderungen in Echtzeit anzeigen.

\subsection{Open-source IoT platform ThingsBoard}
\label{sec:appendix:thingcloud}
ThingsBoard\footnote{https://thingsboard.io} ist eine Open-Source-IoT-Plattform zur Erfassung, Verarbeitung, Visualisierung und Verwaltung der Daten. Es ermöglicht Gerätekonnektivität über Industriestandard-IoT-Protokolle - MQTT, CoAP und HTTP - und unterstützt sowohl Cloud- als auch lokale Bereitstellungen. ThingsBoard kombiniert Skalierbarkeit, Fehlertoleranz und Hardware- und Software Leistungen. \\
Mit ThingsBoard kann man umfassende IoT-Dashboards für die Datenvisualisierung und Remote-Gerätesteuerung in Echtzeit erstellen. Während der Entwicklung des Projekts \textit{"Noise Detector"} die Leistungen von ThingsBoard wurden dafür verwendet, um die Einstellungen auf entwickelten Gerät zu speichern. Das wurde erfolgreich mithilfe JSON-Datei realisiert, deren Format in Abschnitt \ref{sec:appendix:json} auf Seite \pageref{sec:appendix:json} kurz erklärt wird. \\
ThingsBoard ist unter Apache License 2.0 lizenziert, so dass man es kostenlos in seinem kommerziellen Produkten verwenden kann. Es hat auch eine wichtige Rolle für das Projekt gespielt, da \textit{"Noise Detector"} ohne Finanzierung für eigenes Geld des Unternehmens entwickelt wurde. Die Abbildung zeigt ein Beispiel der Verwendung von ThingsBoard für Visualisierung\footnote{http://gluelogics.com/widgets-bundles}. 
\begin{figure}[!htb]
  \centering
   \fbox{\includegraphics[width=1\textwidth]{gfx/thingboard.png}}
   \caption[ThingsBoard Beispiel]{ThingsBoard Beispiel: Maps Widgets}
   \label{fig:thingboard}
\end{figure}

\section{JSON-Datei}
\label{sec:appendix:json}
JSON\footnote{https://www.json.org/} ist die Abkürzung für JavaScript Object Notation, was auf Deutsch bedeutet "JavaScript Objekt-Bezeichnung". Eine JSON-Datei speichert Daten ähnlich wie CSV-, Javascript oder XML-Datei. Sie ist einfach für Menschen zu lesen und zu schreiben. Sie ist einfach für Maschinen zu parsen (Analysieren von Datenstrukturen) und zu generieren. Diese Eigenschaft spielt eine wichtige Rolle, weil sie viel den Arbeitsaufwand für Menschen ersparen kann, womit die Daten in verschiedenen Formaten und aus unterschiedlichen Schnittstellen schnell und leicht zu verarbeiten sind. Dank dieser Eigenschaft in der Zeit des Internets der Dinge wird die Auszeichnungssprache XML zunehmend von einfacheren Formaten wie z. B. JSON ersetzt. 
ThingsBoard arbeitet mit JSON-Datei und die Datei-Format wurde für das Projekt meines Praktikums verwendet, um die Einstellungen auf unseren IoT-Gerät zu ändern. Mithilfe der JSON-Datei kann der Benutzer die Empfindlichkeitsgrenze des Geräts ändern. So kann man ein Gerät entweder für die Bemessung und Feststellung von leisen Geräusche oder für die Arbeit in sehr lauten Räumen so einstellen, dass nach der Überschreitung des Schwellenwerts ein automatischer Alarm ausgelöst wird. Mit JSON-Datei kann man zusätzlich einen von mit programmierten Warnsummer ein- oder ausschalten.  
\subsubsection{Beispiel einer JSON Datei}
\begin{lstlisting}
"title": "Noise Detector Dashboard",
"configuration": {
	"description": "ESP32",
	"name": "Noise Detector Dashboard"
	"widgets": {
		"settings": {
		"stateControllerId": "entity",
		"showTitle": false,
		"showDashboardsSelect": true,
		"showEntitiesSelect": true,
		"showDashboardTimewindow": true }}}  

\end{lstlisting}

\section{SPI Bus}
Das Serial Peripheral Interface, kurz SPI ist ein Bussystem,das aus drei Leitungen für eine serielle synchrone Datenübertragung besteht:
\begin{itemize}
	\item  MOSI (Master Out -> Slave In);
	\item  MISO (Master In <- Slave Out);
	\item  SCK (Serial Clock) - Schiebetakt
\end{itemize}
Zusätzlich zu diesen drei Leitungen benutzt man für jeden Slave eine Slave Select (SS) oder auch Chip Select (CS) genannte Leitung, durch die der Master den Slave zur aktuellen Kommunikation selektiert.
\section{PWM}
PWM steht für (engl.) Pulse Width Modulation und heißt eigentlich übersetzt Pulsbreitenmodulation. PWM wird zum Erzeugen eines analogen Signals anhand einer digitalen Signalquelle benutzt. Ein pulsweitenmoduliertes Signal (PWM-Signal) ist durch zwei Hauptkomponenten charakterisiert: Tastverhältnis und Frequenz. Das Tastverhältnis ist das Verhältnis der High-Dauer des Signals zur Gesamtdauer des Impulses. Die Frequenz bestimmt, wie schnell eine Periode durchlaufen wird (1000 Hz entspräche z. B. 1000 Perioden pro Sekunde), und gibt somit an, wie schnell das Signal zwischen High und Low wechselt. Die Abbildung\footnote{http://www.electronicwings.com/pic/pic18f4550-pwms} zeigt wie unterschiedlich ein Signal aussehen kann, wenn man Tastverhältnis ändert. 
\begin{figure}[!htb]
	\centering
	\fbox{\includegraphics[width=1\textwidth]{gfx/pwm.png}}
	\caption{PWM Signale mit unterschiedlichem Tastverhältnis}
	\label{fig:pwm}
\end{figure}

\section{MQTT Protokoll}
MQTT (MQ Telemetry Transport oder Message Queue Telemetry Transport) wurde 1999 zur M2M-(Maschine zu Maschine)-Kommunikation im Zuge eines gemeinsamen Projekts von IBM und Arcom Control Systems entwickelt. Die Internet Assigned Numbers Authority (IANA) reserviert für MQTT die Ports 1883 und 8883. MQTT ist ein Client-Server-Protokoll. Clients senden dem Server nach Verbindungsaufbau Nachrichten mit einem Topic, die mit dem TLS-Protokoll verschlüsselt werden können. 
%=======================================================

\chapter{Abkürzungverzeichnis}
\label{sec:abkuerz}
\begin{acronym}[SEPSEP]
	\acro{BOM}{Bill of Materials}
	\acro{DEVKIT}{Development Kit}
	\acro{Cloud}{Cloud Computing}
	\acro{EKG}{Elektrokardiogramm}
	\acro{GmbH}{Gesellschaft mit beschränkter Haftung}
	\acro{I$^{2}$S}{Inter-IC Sound}
	\acro{IoT}{Internet of Things, Internet der Dinge}
	\acro{IT}{Informationstechnik, Bereich der Informations- und Datenverarbeitung}
	\acro{JSON}{JavaScript Object Notation}
	\acro{LED}{Light-emitting diode, Leuchtdiode}
	\acro{MQTT}{Message Queuing Telemetry Transport}
    \acro{NVS}{Non-volatile storage}
    \acro{PCB}{Printed circuit board}
	\acro{PWM}{Pulse Width Modulation}
	\acro{SPI}{Serial Peripheral Interface Bus}
	\acro{UV}{Ultraviolettstrahlung}
	
\end{acronym}

%========================================================

\chapter{Abbildungsverzeichnis}
\label{sec:abbild}
\renewcommand{\cftfigpresnum}{Abb. } 
\renewcommand{\listfigurename}{}
\setlength{\cftfignumwidth}{2 cm}
\listoffigures{}% Abbildungsverzeichnis

