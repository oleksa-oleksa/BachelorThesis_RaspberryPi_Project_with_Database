\chapter{Fazit}
\section{Praktikum und Studium}
Die im Bachelorstudiengang Technische Informatik erworbenen Grundlagen und Fähigkeiten ermöglichten es mir, Aufgaben im Praktikum zu erfüllen. Besonderer Fokus lag hier auf den praktischen Erfahrungen mit den Programmiersprachen C und C++, Analog- und Digitaltechnik, sowie grundlegenden Problemlösungsstrategien und die Fähigkeiten mit dem neuen Stoff umzugehen und die Lösungen der Probleme zu finden, die ich während meines Studiums nicht betrachtet habe.

Ein Beispiel hierfür ist die Beherrschung der Software \textit{EAGLE} für das Design und Fertigstellung des Schaltplans, Aufbau der Platine und Festlegung der Komponenten in der Datei namens "Bill of Materials".

Von den Modulen, die ich an der Beuth Hochschule schon absolviert habe, besonders hilfreich für meinen Praktikumsablauf waren \textit{Systemprogrammierung, Maschinenorientierte Programmierung, Analoge Elektronik (Elektrische Systeme III)} und \textit{Mikrocomputertechnik}. Da auf ESP32 FreeRTOS läuft, habe ich viele Kenntnisse aus dem Modul \textit{Echtzeitsysteme} in der Realität verwendet. Letztendlich wären die Aufgaben, die der von mir programmierter \textit{Noise Detector} ausführen konnte, ohne mehreren parallelen Tasks nicht möglich. Obwohl während des Studiums FreeRTOS nicht genau betrachtet wurde, konnte ich leicht nachvollziehen, wie man die Tasks in der RTOS Umgebung programmiert. 

Während meines Praktikum habe ich festgestellt, dass die Programmierung mir den größten Spaß macht und der Schwerpunkt meiner zukünftigen Arbeit wird. Hier wurde mein Interesse für die Embedded Programmierung unterstützt und die Neugier für die wachsende Branche der IoT-Geräte geweckt.

\section{Bewertung des Praktikums}

Mein Praktikum bei Eptecon hatte eine angenehme, offene Atmosphäre. Als Praktikantin kriegte ich alle meine Fragen von meinem Praktikumsleiter und anderen Mitarbeitern beantwortet. Ich fühlte mich nie unter Druck gesetzt etwas einfach nur schnell abarbeiten zu müssen, sondern es wurde viel Wert auf Verständnis und das Erlernen neuer Fähigkeiten gelegt. 

Ein Beispiel hierfür ist, dass mir die Zeit gegeben wurde, mich ganz genau Zeit mit der Verwendung der Software \textit{EAGLE} zu befassen. Am Anfang meines Praktikums hatte ich auch entsprechend genug Zeit, um alle Datenblätter durchzulesen, Tool Chain zu installieren und mich mit dem Mikrocontroller ESP32 vertraut zu machen. 
 
Es war eine sehr anregende Zeit, in der ich sehr viel lernte und viele neue Fachgebiete kennenlernte. Ich habe mich gefreut, dass ich auch schon viel an der Beuth Hochschule gelernt habe, was man in der Praxis bei einer realen Arbeit verwenden kann und muss. Ich weiß jetzt genau, worin ich die Vertiefung machen will. 
