% !TeX spellcheck = en_GB 
\chapter{Technical basis}
\label{sec:main}
\section{AcaLoan-Raspi Project of the PSE laboratory}
\label{sec:main:overview}
Meine Aufgabe während meines Praktikums war die Programmierung 

\subsection{Mikrocontroller ESP32}
Die Aufgaben und Programmierung des Mikrocontrollers ESP32 haben mich mit einem neuen Konzept vertraut gemacht.

\subsection{Click Boards von Mikroelektronika}
\begin{wrapfigure}{l}{0.4\textwidth}
	\centering
	\fbox{\includegraphics[width=0.3\textwidth]{gfx/noise.png}}
	\caption{Noise Click Board}
	\label{fig:click}
\end{wrapfigure}
MikroElektronika\footnote{https://www.mikroe.com} ist ein Hersteller und Händler von Hardware- und Software-Tools für die Entwicklung eingebetteter Systeme. 

\section{Vorbereitung}
\label{sec:main:preparation}
Beim ersten Gespräch mit meinem Praktikumsbetreuer Herrn Prizkau informierte ich mich, welche Programmierumgebungen und Programmiersprachen für \textit{"Noise Detector"} Projekt verwendet werden.
Ebenfalls erkundigte ich mich nach der Dokumentation und Dattenblättern der verwendenden Hardware.
\begin{figure}[!hb]
	\centering
	\fbox{\includegraphics[width=1\textwidth]{gfx/esp32_entwick.jpg}}
	\caption[Praktikum Projekt]{Praktikum Projekt: Breadboard mit ESP32, Noise Click und Buzz 2 Click Boards}
	\label{fig:breadboard}
\end{figure}
Als Hardware für meine Programmierungsaufgaben habe ich eine Steckplatine bekommen, auf der mir zur Verfügung ESP32 DEVKIT von Espressif, Noise Click und zwei Clicks Boards von Mikroelektronika sich befanden.

\section{Aufgaben}
Meine Aufgaben waren in folgende Teilbereiche gegliedert:: 
\begin{itemize}
	\item Den Schwellenwert für Noise Click Board berechnen und ihn durch SPI Bus entsprechend programmieren. 
	\item Den Summer durch PWM steuern, einschalten und ausschalten.{\normalsize }
	\item Die Interrupt Service Routine entwickeln, in der ESP32 Board die ankommenden Interrupts bearbeiten und entsprechen reagieren kann:
	\subitem * einen automatischen Alarm mit Hilfe von Summer erzeugen.
	\subitem * einen LED anschalten.
	\subitem * eine Datei auf Cloud Server mit Alarm-Werten schicken.
	\item Wi-Fi  programmieren, JSON-Datei für Input/Output Daten erstellen.
	\item von Cloud Server die entsprechenden JSON Dateien lesen, bearbeiten und Einstellungen des \textit{Noise Detectors} entsprechend der Wünschen des Benutzers ändern.
	\item Anbindung Sensoren/Aktoren an Mikrocontroller testen
	\item Fertigstellung des Schaltplan mithilfe \textit{EAGLE-}Software, Fertigung des Layouts, Bestellung der Platine.
	\item Kurze Projektdokumentation erstellen.	
\end{itemize}