\chapter{Einleitung}
\label{sec:intro}
\section{Vorstellung des Praktikumsbetriebes}
\label{sec:intro:dfki}
Eptecon GmbH gehört zu den Berliner Startup-Szene und wurde in dem Jahr 2016 gegründet. Obwohl das Unternehmen kürzlich gegründet wurde, kann man schon feststellen, dass die erste Phase des Lebenszyklus erfolgreich war und Eptecon ein paar Projekte schon abgeschlossen hat. \\
Das Unternehmen haltet sich auf einem Ortsteil am Ostrands des Bezirks Spandau namens Siemensstadt auf, die durch das Wachstum der Werke von Siemens \& Halske entstand. 
Eptecon bietet Dienstleistungen in den Bereichen der Hardware und Embedded Software Entwicklung für die Anbindung von Sensoren, Geräten und Systemen an das Internet, der Engineering und der Integration von Connectivity-Lösungen sowie der Technologieberatung für Startups und kleine und mittelständische Unternehmen an. Mit einem Team von Ingenieuren und Technik-Enthusiasten mit über 15 Jahren Berufserfahrung in den Feldern der industriellen Prozessinstrumentierung, Medizintechnik und Funkkommunikation sowie studentischen Mitarbeitern erforscht und entwickelt Eptecon GmbH IoT-Anwendungen, IoT-Produkte und IoT-Lösungen. Eptecon unterstützt Unternehmen bei zusätzliche Entwicklungsarbeit, die benötigt wird, um die physische Welt mit IoT-Plattformen zu verbinden. \\
Als ein Startup hat aber Eptecon bisher mehrere Projekte erfolgreich abgeschlossen.
\begin{itemize}
	\item \textbf{Röst-Mahl-Koch-Kaffeemaschine:} Integration von Connectivity und Entwicklung von IoT-Services für Röst-Mahl-Koch-Kaffeemaschine für Bonaverde\footnote{https://www.bonaverde.com/}. Sie nutzt IoT für das Sammeln von Gerätedaten und die Steuerung des Kaffeeherstellungsprozesses. Dies ermöglicht Fernwartung, Neubestellung und Bezahlung von Kaffeebohnen sowie neuartige Benutzererfahrung durch Integration eines Messenger Bots.
	\item \textbf{iPhone Add-on für EKG Messung:} Technologieberatung und Vorbereitung der Massenproduktion eines Add-ons für einfache Messung von Elektrokardiogrammen (EKG) für CardioQvark\footnote{http://www.cardioqvark.ru}. Das EKG-Mess-Add-on ermöglicht die Erfassung von EKG mit nur zwei Fingern. Das aufgezeichnete EKG wird dann zur automatischen Analyse und Verarbeitung an die Cloud geschickt. Die Ergebnisse werden in einer iOS-Anwendung angezeigt und können sehr einfach mit dem zuständigen Arzt geteilt werden. 
	\item \textbf{Wearable UV-Sensor für Sonnenbrandprävention:} Technologiebewertung und Systemdesign eines tragbaren Produktes zur Messung der Sonneneinstrahlung für UVizr\footnote{http://www.uvisio.com}. Ein kleines, tragbares Accessoire mit UV-Sensor warnt seinen Besitzer vor möglichem Sonnenbrand. Eine Smartphone-Anwendung kommuniziert die UV-Empfindlichkeit der Haut des Benutzers an den Sensor und erhält UV-Messdaten zur weiteren Verarbeitung über Bluetooth. 
	\item \textbf{Iot-Plattform für LED-Beleuchtung:} Systemdesign und Entwicklung einer modularen Plattform für die Verbindung von intelligenter Beleuchtung mit dem Internet der Dinge für lumilabs.\footnote{http://www.lumilabs.de}. Die Plattform kombiniert Benutzer- und Sensor-basierte LED-Leuchtensteuerung mit drahtloser und drahtgebundener Kommunikation sowie Datenanalytik. Es lässt sich problemlos in nahezu jede Leuchte integrieren und ermöglicht, neben der Lichtsteuerung, auch das Erfassenen und Bereitstellen von Umgebungsdaten. 
\end{itemize}  
Ich absolvierte mein Praktikum innerhalb des neuen Projekts \textit{"Noise Detector"}, das ein internes Hardware-Projekt des Unternehmens sein sollte, mit dem Eptecon seine eigenen IoT-Geräte entwickeln wollte. Meine Arbeit wurde durch Dipl.-Ing. Erwin Prizkau geleitet.
\\\\
Meine Aufgabengebiete konzentrierten sich auf die Programmierung der kleinen System-Entwicklungs-Board ESP32 DEVKIT\footnote{https://www.espressif.com/en/products/hardware/esp32/overview}. Als Programmiersprache wurde C-Programmiersprache verwendet, was besonders interessant für mich ist, da in meinem Studiengang \textit{"Technische Informatik"} diese Sprache als erste und primäre Sprache gelehrt wird und ich über gute Kenntnisse dieser Programmiersprache verfüge und eine praktische Erfahrung möglich bald bekommen wollte. Nach erfolgreiche Programmierung des Boards sollte ich an der Entwicklung des Hardwares von neuen Gerät teilnehmen und ganz neue Kenntnisse über sowohl Fertigstellung des Schaltplans, Aufbau der Hardware als auch über die Anbindung von Sensoren und Aktoren an Mikrocontroller bekommen.
Das Projekt \textit{"Noise Detector"} wird in Abschnitt \ref{sec:main:overview} auf Seite \pageref{sec:main:overview} näher erläutert.
\section{Weg zur Praktikumsstelle}
\label{sec:intro:wegZurPraktikumsstelle}
Seit dem Jahr 2012 habe ich ein Facebook Konto und bin seit ein paar Jahren aktives Mitglied einer Facebook Community namens <<IT Berlin>>\footnote{https://www.facebook.com/groups/itberlin/}. Die Facebook Seite ist für russischsprachige Softwareentwickler und Softwareingenieur gedacht, die in Berlin wohnen oder nach Berlin umziehen wollen. Die Seite ermöglicht es den Mitgliedern, sich über verschiedene Aspekte des Lebens und der Arbeit in Berlin auszutauschen. Die Facebook Community wird von Arbeitnehmer und Headhuntern benutz, um die Mitglieder über die neuen Jobangebote zu informieren und neue Kollege zu suchen. Ich komme aus der Ukraine und verfüge über zwei Muttersprachen: Ukrainisch und Russisch, deshalb finde ich <<IT Berlin>> ziemlich interessant für mich und meine zukünftige Arbeit. \\\\
Weil ich nach 3. Fachsemester 100 Kredits gesammelt habe, wurden alle Voraussetzungen für Bewerbung für ein Praktikum im Studium erfühlt. Da <<IT Berlin>> eine sehr aktive lebendige Community ist, habe ich früher die verschiedene Kommentaren von Herrn Prizkau gesehen und festgestellt, dass er in einem Unternehmen arbeitet, das sich mit Software Lösungen für Embedded Systems und IoT-Anwendungen beschäftigt. Als ich mich entschieden habe, in meinem 5. Fachsemester in die Praxisphase zu gehen, habe ich obengenanntem Mitglied eine Freundschaftsanfrage geschickt und eine Nachricht geschrieben, in der ich mitgeteilt habe, dass ich eine Studentin von Beuth Hochschule für Technik bin und gerade Technische Informatik mit dem Schwerpunkt <<Embedded Systems>> studiere. Herr Prizkau hat Freundschaftsantrag akzeptiert und mir mitgeteilt, dass in seinem Unternehmen Bedarf und Interesse an Praktikanten und studentischen Mitarbeitern besteht.\\\\
Ich habe mich daraufhin auf der Website des Unternehmens Eptecon\footnote{http://eptecon.de} über das Unternehmen und seine Projekte informiert. Da ich mich sehr für Programmierung des Hardware und die Entwicklung der Embedded Software interessiere, fand ich die Möglichkeit meine Praxisphase bei Eptecon zu absolvieren sehr interessant. Nach einem persönlichen Gespräch mit Herrn Prizkau, in dem ich meine Kenntnisse in Mikroelektronik und Programmierung nachweisen und mein Interesse für das Projekt darlegen konnte, kam es zur Vertragsunterzeichnung.