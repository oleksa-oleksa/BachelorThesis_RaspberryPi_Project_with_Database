% !TeX spellcheck = de_DE
\chapter{Analyse}
\label{sec:analyse}
Bei der Entwicklung benutzerdefinierter Webanwendung "acaLoan" wurde Python einer anderen Programmiersprache bevorzugt, um ihre einfache und ausdrucksstarke Syntax zu nutzen. Da Django in der Programmiersprache Python geschrieben ist, wird es auch sinnvoller, die Abschlussarbeit in Python Code zu erstellen, indem die alle Komponenten Syntaxregeln von Python nutzen. Django stellt die Ressourcen bereit, die währen der Implementierung der analysierte Software sofort benötigt wurden: allgemeine Vorgänge wie Datenbankmanipulation, HTML-Vorlagen, URL-Routing, Sitzungsverwaltung und Sicherheit. Mit der MVC-Prinzip (Model-View-Controller) sind die Benutzeroberfläche (UI) und die Geschäftslogikebenen des acaLoan-Systems getrennt. Im Gegensatz zu anderen modernen Webframeworks erlaubt Django einzelnen Prozessen nicht, mehrere Anforderungen gleichzeitig zu verarbeiten. Zwar gilt es oft als ein Nachteil, ist es aber eine notwendige Eigenheit für das acaLoan-System, da es nur eine aktive Benutzersitzung erlaubt ist. Django unterstützt verschiedene Betriebssysteme wie Windows, Linux und MacOS, damit es keine strenge Softwareabhängigkeiten bei acaLoan-System während sowohl der Bereitstellung des Djangos mit ASGI als auch in der Zukunft gegeben werden sollten. 

Für sowohl eine Markierung der Raspi-Boards als auch für eine Authentifizierung der Studierenden werden RFID-Tags benutzt. Da RFID Funkwellen zur Kommunikation verwendet werden, müssen sich RFID-Tags nur innerhalb des Lesebereichs des Lesegeräts befinden. Das lässt die kleine NFC-Tags auf der Rückseite des jeden Boards unter dem Schutzschirm zu kleben, sodass der Board am RFID-Leser ohne präzise Positionierung auf dem Lesefeld angelesen werden kann. Da die Studentenkarte der Beuth Hochschule mit MIFARE DESFire kontaktlose Chipkartentechnik hergestellt sind, wird es möglich, in acaLoan-System nur mit einem RFID-Leser zu arbeiten und sowohl die Daten der Studierende als auch die Raspi-Boards abzulesen. 

An dieser Stelle muss es gesagt werden, dass die RFID-Tags bestimmte Nachteile haben, die wurden zwischen den Mitarbeiter und Autorin der Abschlussarbeit diskutiert. Es wurde jedoch die Entscheidung getroffen, dass RFID-Tags für die vorgesehenen Zwecken des Verfolgen des Raspi-Boards (welchen Raspi-Board von welchem Student am welchen Tag ausgeliehen wurde und bis zum welchen Tag zurückgegeben muss) die Sicherheitserwartungen der PSE-Labor Mitarbeiter erfühlen. RFID-Tags sind aus mehreren Gründen im Vergleich nicht ideal. Da ein RFID-Tag nicht zwischen Lesegeräten unterscheiden kann, können die Informationen von fast jedem gelesen werden, sobald sie die ursprüngliche Lieferkette verlassen haben. Weil RFID-Lesegeräte so tragbar sind und die Reichweite einiger Tags so groß ist, können Betrüger Informationen sammeln, auf die sie sonst keinen Zugriff hätten. Dies bedeutet, dass jeder ohne Wissen einer Person potenziell sensible Informationen sammeln kann. Diese Nachteile der RFID-Tags wurden vernachlässigt, da sowohl Studentenkarte als auch geklebte auf den Raspi-Boards RFID-Tags keine sensible Information behalten und eine kleine Reichweite von bis zu 10 cm haben. Es gibt trotzdem ein Gefahr, dass entweder die Studentenkarte geklont von einem Täter wird, um sich für einen Student ausgeben und ein Raspi-Board stehlen zu können, oder ein Raspi-Board geklont wird, um ein Datensatz in der Datenbank zu erzeugen, dass schon ausgeliehenen Board quasi zurückgegeben wurde, obwohl in der Realität den Raspi-Board nie ins Labor zurückgekommen war. Gegen das Klonen des Raspi-Tags wurde es besprochen, dass in der Zukunft im PSE-Labor im Schrank mit den Raspi-Boards jeden Platz für jeden entsprechenden Raspi-Board mit einem Gewichtssensor ausgerüstet werden wird und acaLoan-System auf der Erscheinung des bestimmten Gewicht erwarten wird. Dies ist aber nicht der Teil bestehenden Abschlussarbeit und von Mitarbeiter des PSE-Labor als eine spannende Aufgabe für die andere Abschlussarbeit vorgesehen ist. Gegen das Klonen des Studentenkarte wurde es zuerst entschieden, dass ein bestehenden Zugang zu einem Schrank mit Raspi-Boards entlang die beide Arbeitstischen der Mitarbeiter des PSE-Labor eine bestimmte Sicherheit gewährleisten könnte. Die anderen Lösungen werden nach der Lieferung der Software diskutieren und liegen ebenfalls außer den Rahmen der bestehenden Abschlussarbeit. 

Obwohl oben die Nachteilen der RFID-Tags erwähnt wurde, es lässt sich zusammenfassend sagen, dass die neue Studentenkarte, die an der Beuth Hochschule ab Sommersemester 2018 verwendet wurden, sind eine zuverlässige und zeitgemäße Lösung. Die Karte beinhaltet keine elektronischen persönlichen Daten der Studierenden und die Campus-Automaten alle persönlichen Daten anhand eines Pseudonyms online abrufen müssen (d.h. liegen in den Automaten auch keine persönlichen Daten vor). Sodass im Fall des Verlusts die persönliche Daten von den Unberechtigte nicht ausgelesen werden können \cite{website:12}. Das stand im Fokus der Entscheidung, eine Studentenkarte als einzige elektronischer Identifizierungsmittel beim Ausleihe/Rückgabevorgänge im PSE-Laboz zu benutzen.

In dieser Abschlussarbeit wurde nachgewiesen, dass das Verleihprozedere für die Raspi-Boards (Lab und Home) mit gewählten Mittel für die Verwendung im PSE-Labor erfolgreich automatisiert wurden. 

\chapter{Zusammenfassung}
\label{sec:results}
Das zu Beginn der Arbeit gesetzte Ziel der Entwicklung von Software für PSE-Labor wurde erfolgreich erreicht. Die Implementierung alle Bestandteilen und ein Zusammenspiel unter der Verwendung des Entwicklungsservers wurde erfolgreich den Mitarbeitern des PSE-Labors präsentiert. Mit dem acaLoan-System werden die Studierende in die Lage versetzt, einen Raspi-Board selbständig im PSE-Labor für die Übungen auszuleihen und später zurückzugeben. Damit wird ein Arbeitszeitverbrauch für die Verwaltung der Bordstandortbestimmung reduziert und die Mitarbeitern können sich auf weitere neuen geistliche wissenschaftlichen Herausforderungen konzentrieren und damit in der Weiterentwicklung des acaLab-Projekte hineinbringen.  Der Funktionsumfang der acaLoan-System erlaubt es den Studierenden, den Raspberry Board entweder für die Übung im Labor oder für die selbständige Arbeit zu Hause auszuleihen. 

Darüber hinaus wird der Kommunikation der Studierende mit dem acaLoan-System auf notwendigen Minimum reduziert, sodass wird ein Ausleihe- / Rückgabevorgang nur mit drei Tastendruck erledigt. Zuerst wird die Taste "Loan / Return Board" gedruckt, um eine neue Benutzersitzung zu starten. Das Ablesen der Studentenkarte und Raspi-Boards geschieht ohne weitere Tastendrucken nur über das Register-Client, der die abgelesene Daten dem Server in den Endpunkt der REST API schickt. Es wird abhängig von dem Typ des Boards bestimmt, ob abgelesenen Board zum Ausleihen oder Rückgabe ist. Mit dem zweiten Tastendruck bestätigt der Studierende seinen Wunsch, der Board auszuleihen oder zurückzugeben. Mit dem dritten Tastendruck nach dem erfolgreichen Beenden der Operation, enden der Studierende seine Sitzung und gibt acaLoan-System dem nächsten Studierenden wieder frei. 

Für die Implementierung acaLoan-System ist ein ausführliches Software Engineering in Form der objektorientierten Analyse und des Designs der endlichen Zustandsmaschine durchgeführt worden. Das acaLoan-System konnte ohne Server nicht realisiert werden, da Benutzerdaten länger als eine Sitzung gespeichert werden müssen: die Studentenkarten und die Datensätze alle Ausleihvorgänge müssen mindestens für ein laufenden Semester gespeichert werden, damit die Mitarbeiter des PSE-Labor immer einen Zugang zu allen gespeicherten vorherigen Leihvorgangs (von der Ausleihe bis zur Rückgabe eines Boards) bekommen können. Der Server ist erfolgreich mit Django Webframework realisiert. Es ist auch vorgesehen, dass am Ende des Semester nach dem letzte Rückgabe eines Boards die Datensätzen des zu Ende gegangen Semesters gelöscht werden. Der Zugriff von Server auf Display-Client wird durch entsprechende Schnittstellen mithilfe der RESTful Kommunikation ermöglicht. In der realisierten Abschlussarbeit wurde auch ein notwendigen Teil namens Register-Client entwickelt, an dem ein RFID-Leser angeschlossen wurde. Der Register-Client verfügt selbst über keinen Datenbank und darf nur die abgelesene Daten dem Server schicken. Es geht um eine Simplex-Verbindung, da ein Nachrichtenverkehr asymmetrisch ist, weil der Register-Client keine Daten vom Server zurückbekommen darf und über den erfolgreiche oder gescheiterte Leihvorgang nicht wissen muss. Für die Implementierung des Register-Clients wurde uComputer Raspberry Pi benutzt, der schon zur Verfügung im PSE-Labor stand und ergänzend nicht geliefert werden muss. 