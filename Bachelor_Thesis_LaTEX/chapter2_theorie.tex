% !TeX spellcheck = de_DE
\chapter{Der theoretische Hintergrund}
\label{sec:theorie}
Als schon im Abschnitt "Einleitung" erwähnt wurde, bei der Aufgabestellung es um eine Entwicklung einer Webanwendung geht. Die zu realisierende Software basiert sich, wie die meisten Webanwendungen, auf einer Client-Server-Architektur, wobei der Client Informationen eingibt, während der Server die eingegebene Daten empfängt, bearbeitet und speichert. Eine Webanwendung ist ein Computerprogramm, das eine bestimmte Funktion unter Verwendung eines Webbrowsers als Client ausführt. Die Webanwendung kann so einfach wie ein Kontaktformular auf einer Website oder so komplex wie eine Textverarbeitungs- oder Bildbearbeitungsprogramm sein, die Sie auf Ihr Computer im Browser ausführen können. Um die Webanwendung zu starten, muss der Benutzer keine zusätzlichen Programme installieren. Sie wird auf jedem Gerät mit Browser und Internetzugang ausgeführt. Der Client ist nicht vom Betriebssystem auf dem Computer des Benutzers abhängig. Bei der Entwicklung von Webanwendungen müssen daher keine separaten Versionen für Windows, Linux, Mac OS und andere Betriebssysteme geschrieben werden. Zur Implementierung der Clientseite werden HTML, CSS, JavaScript und Ajax verwendet. Zum Erstellen der Serverseite von Webanwendungen werden Programmiersprachen wie PHP, ASP, ASP.NET, Perl, C / C ++, Java, Python, Ruby und NodeJS verwendet. In dem vorherigen Schritt wurde es die Entscheidung getroffen, die Serverseite mit Web-Framework Django zu erstellen. Django wird mit einem leichten Webserver geliefert, mit dem eine  Website schnell zum Laufen gebracht werden kann, ohne Zeit mit der Einrichtung eines Servers verschwenden zu müssen. Wenn der Entwicklungsserver von Django gestartet wird, überwacht er die Codeänderungen in Echtzeit. Es wird nach dem Ändern des Codes automatisch neu geladen.

Da in den letzten Jahren sich Webanwendungen rasant weiterentwickelt und die Desktop-Lösungen schrittweise ersetzt haben und sind zu einem wesentlichen Bestandteil des Geschäfts in der modernen Welt geworden haben, es sollte auch nicht unerwähnt bleiben, welche Vorteile die Webanwendungen haben:

\begin{itemize}
	\item \textbf{Zugriff von jedem Gerät} Die Webanwendung kann überall auf der Welt von einem Computer, Tablet oder Smartphone zugegriffen und verwendet werden. Notwendig ist, dass dem Gerät eine Internetverbindung zur Verfügung steht.

\item \textbf{die Kostenersparnis} Webanwendungen können auf allen Plattformen ausgeführt werden und müssen nicht mehr separat für Android und iOS entwickelt werden.

\item \textbf{Anpassungsfähigkeit} Wenn native Anwendungen bestimmte Betriebssysteme erfordern, jedoch können jedes Betriebssystem (Windows, MAC, Linux usw.) und jeder Browser (Internet Explorer, Opera, FireFox, Google Chrome usw.) für die Arbeit mit einer Webanwendung. ) verwendet werden.

\item \textbf{Keine Software zum Herunterladen} Günstig und einfach dem Endnutzer zu liefern, zu warten und zu aktualisieren. Das Aktualisieren auf die neueste Version erfolgt beim nächsten Laden der Webseite.

\item \textbf{Netzwerksicherheit} Das Websystem verfügt über einen einzigen Einstiegspunkt, der zentral geschützt und konfiguriert werden kann.

\item \textbf{Skalierbarkeit} Mit zunehmender Belastung des Systems ist es nicht erforderlich, die Leistung des Computer von Endbenutzer zu erhöhen. Mit einer Webanwendung kann in der Regel nur mithilfe von Hardwareressourcen eine größere Datenmenge verarbeitet werden, ohne den Quellcode neu zu schreiben und die Architektur ändern zu müssen.

\item \textbf{Verhinderung von Datenverlust} Benutzerdaten werden in der "Cloud" gespeichert, für deren Integrität die Hosting-Anbieter verantwortlich sind, deswegen vom Verlust geschützt, falls die Festplatte des Computers beschädigt wird.
\end{itemize}

\section{Über Raspberry Pi Board und OS}
\label{sec:theorie:raspberry}

\section{Kontaktlose Chipkartentechnik MIFARE}
\label{sec:theorie:mifare}

\section{Sender-Empfänger-System mit RFID}
\label{sec:theorie:rfid}

\section{Datenbanken mit Python und SQLite}
\label{sec:theorie:db}

\section{HTTP für Design der verteilten Systeme}
\label{sec:theorie:http}

\section{Django Framework}
\label{sec:theorie:about_django}

\section{API}
\label{sec:theorie:api}

\section{Endliche Zustandsmaschine}
\label{sec:theorie:fsm}

\section{Clientseitiges JavaScript}
\label{sec:theorie:js}




