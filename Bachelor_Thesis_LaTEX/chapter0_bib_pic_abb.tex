% !TeX spellcheck = de_DE
\chapter*{Abbildungsverzeichnis}
\label{sec:list_img}
\addcontentsline{toc}{chapter}{\protect\numberline{}Abbildungsverzeichnis}%
\renewcommand{\cftfigpresnum}{Abb. } 
\renewcommand{\listfigurename}{}
\setlength{\cftfignumwidth}{2 cm}
\listoffigures{}% Abbildungsverzeichnis

%========================================================
\chapter*{Abkürzungsverzeichnis}
\label{sec:list_abbr}
\addcontentsline{toc}{chapter}{\protect\numberline{}Abkürzungsverzeichnis}%
\begin{acronym}[SEPSEP]
	\acro{AJAX}{Asynchrones Javascript und XML}
	\acro{API}{Application Programming Interface}
	\acro{ASGI}{Asynchronous Server Gateway Interface}
	\acro{ATR}{Answer-To-Reset}
	\acro{BGA}{Ball Grid Array}
	\acro{CPU}{Central processing unit }
	\acro{DOM}{Document Object Model}
	\acro{GPU}{Graphic processing unit }
	\acro{HTTP}{HyperText Transfer Protocol}
	\acro{JSON}{JavaScript Object Notation}
	\acro{IDE}{Integrated Development Environment}
	\acro{IEEE}{Institute of Electrical and Electronics Engineers}
	\acro{ISO}{International Organization for Standardization}
	\acro{MAC}{Media Accesss Control Layer}
	\acro{NFC}{Near Field Communication}
	\acro{SDRAM}{Synchronous dynamic random-access memory}
	\acro{SoC}{System-on-Chip}
	\acro{RAM}{Random-access memory}
	\acro{Raspi}{Raspberry Pi Board und Minicomputer im PSE-Labor der BHS}
	\acro{RFID}{Radio-frequency identification}
	\acro{URI}{Uniform Resource Identifier, Unified Resource Identifier}
	\acro{URN}{Uniform Ressource Name}
	
\end{acronym}





