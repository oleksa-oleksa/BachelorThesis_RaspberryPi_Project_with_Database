% !TeX spellcheck = de_DE
%\appendix
\chapter*{Glossar}
\addcontentsline{toc}{chapter}{\protect\numberline{Glossar}}%
\section*{AJAX}
\label{sec:appendix:ajax}
AJAX (asynchrones Javascript und XML) ist der allgemeine Name für Technologien, mit denen asynchrone Anforderungen (ohne erneutes Laden von Seiten) an den Server gestellt und Daten ausgetauscht werden können. Da die Client- und Serverteile der Webanwendung in verschiedenen Programmiersprachen geschrieben sind, müssen zum Austausch von Informationen die Datenstrukturen (z. B. Listen und Wörterbücher), in denen sie gespeichert sind, in das JSON-Format konvertiert werden.

\section*{BGA}
\label{sec:appendix:bga}
BGA oder Ball Grid Array ist eine Art oberflächenmontiertes Gehäuse, das in elektronischen Produkten zur Montage integrierter Schaltkreise wie Mikroprozessoren, FPGAs, WiFi-Chips usw. verwendet wird. Die Anschlüsse liegen in Form von Lötkugeln vor, die in einem Gitter angeordnet wie Muster auf der Unterseite des Gehäuses sind, um den für die Verbindungen verwendeten Bereich zu vergrößern.

\section*{DOM}
\label{sec:appendix:dom}
DOM (Document Object Model) ist die Struktur einer HTML-Seite. Bei der Arbeit mit dem DOM werden HTML-Tags (Elemente auf einer Seite) gefunden, hinzugefügt, geändert, verschoben und entfernt.

\section*{HOST}
\label{sec:appendix:host}
Host - Dies  der Name der IP-Adresse für den Webserver, auf den zugegriffen wird. Dies ist normalerweise der Teil der URL, der unmittelbar auf den Doppelpunkt und zwei Schrägstriche folgt.\cite[p.31]{shklar:webapplication} 

\section*{HTML-Vorlage}
\label{sec:appendix:html}
Eine HTML-Vorlage ist eine intelligente HTML-Seite, die Variablen anstelle bestimmter Werte verwendet und verschiedene Operatoren bereitstellt: if (if-then), for-Schleife (Durchlaufen einer Liste) und andere. Das Abrufen einer HTML-Seite aus einer Vorlage durch Ersetzen von Variablen und Anwenden von Operatoren wird als Vorlagenrendering bezeichnet. Die resultierende Seite wird dem Benutzer angezeigt. Falls einen anderen Abschnitt zu öffnen ist, muss ein anderen Musters geladen werden. Wenn andere Daten in der Vorlage verwendet werden müssen, werden sie vom Server angefordert. Alle Formularübermittlungen mit Daten sind auch AJAX-Anforderungen an den Server.

\section*{HTTP}
\label{sec:appendix:http}
HTTP steht für HyperText Transfer Protocol, Hypertext Transfer Protocol". HTTP ist ein weit verbreitetes Datenübertragungsprotokoll, das ursprünglich für die Übertragung von Hypertextdokumenten vorgesehen war (Dokumente, die möglicherweise Links enthalten, mit denen Sie den Übergang zu anderen Dokumenten organisieren können). Die Basis dieses Protokolls ist eine Anforderung von einem Client (Browser) an einen Server und eine Serverantwort an einen Client.

\section*{JSON}
\label{sec:appendix:json}
JSON (JavaScript Object Notation) ist ein universelles Format für den Datenaustausch zwischen einem Client und einem Server. Es ist eine einfache Zeichenfolge, die in jeder Programmiersprache verwendet werden kann.

\section*{PORT} 
\label{sec:appendix:port}
Dies ist ein optionaler Teil der URL, der die Portnummer angibt, die der Zielwebserver abhört. Die Standardportnummer für HTTP-Server ist 80, einige Konfigurationen sind jedoch so eingerichtet, dass sie eine alternative Portnummer verwenden. In diesem Fall muss diese Nummer in der URL angegeben werden. Die Portnummer wird direkt mit einem Doppelpunkt, der unmittelbar auf den Servernamen oder die Adresse folgt, eingegeben.\cite[p.31]{shklar:webapplication} 

\section*{RFID}
\label{sec:appendix:rfid}
RFID (englisch radio-frequency identification oder „Identifizierung mit Hilfe elektromagnetischer Wellen“) bezeichnet eine Technologie für Sender-Empfänger-Systeme und wird bei der Abschlussarbeit verwendet, um die vorhandenen zur Ausleihe Raspberry Pi Boards zu markieren und identifizieren.

\section*{URI}
\label{sec:appendix:uri}
Uniform Resource Identifier ist ein Pfad zu einer bestimmten Ressource (z.B.  einem Dokument), für die eine Operation ausgeführt werden muss (z. B. bei Verwendung der GET-Methode bedeutet dies das Abrufen einer Ressource). Einige Anforderungen beziehen sich möglicherweise nicht auf eine Ressource und in diesem Fall kann der Startzeile anstelle des URI ein Sternchen (Symbol "*") hinzugefügt werden.


 