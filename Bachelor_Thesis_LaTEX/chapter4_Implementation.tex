% !TeX spellcheck = de_DE 
\chapter{Implementierung}
In meiner Abschlussarbeit präsentiere ich die praktische Lösung für das PSE-Labor der Beuth Hochschule für Technik Berlin. Die gesamte Aufgabe lässt sich in drei Bestandteile unterteilen: Register-Client, Server und Display-Client. Erstens wird Register-Client implementiert, damit wird RFID Leser am Raspberry Pi Mikrocomputer angeschlossen, alle Treiber installiert und auf Python Programmierung Sprache die Software geschrieben, die die ständige Überwachung des empfangenden von RFID Leser Daten zulässt und die Verbindung mit dem Server zulässt. Falls die empfangene Daten korrekt sind, d.h.  eine richtige MIFARE Studentenkarte oder einen richtigen RFID-Transponder abgelesen wurde, schickt die Software die abgelesene Daten zum Server ab. Der Server ist der zweite Bestandteil der Abschlussarbeit und wird mit Hilfe Django Framework, Django Finite State Machine auf Python Programming Sprache implementiert. Server enthält die Datenbank mit die Datensätzen über die alle im PSE-Labor vorhandenen ausleihenden Boards, die zum Modul im laufenden Semester registrierten Studenten und  geschehenen Ausleihe/Rückgabe-Vorgänge. Es wird von Server überprüft, ob eine von Register-Client abgelesene Studentenkarte einem zugelassenen für die Ausleihe Student gehört und die entsprechenden Information auf Display-Client geschickt. Es wird auch von Server bestätigt, ob für die Ausleihe/Rückgabe neben dem RFID-Leser gehaltenen Raspi Board dem Student ausgeliehen/vom Student zurückgegeben werden darf. Drittens wird der Display-Client als dynamische HTML-Seite realisiert, die eine Verbindung zum Server Mithilfe des HTTP-Protokolls und eingebauten im Browser Kommunikationsmittel die asynchrone Nachrichten zu schicken, bereitstellt. Für die dynamische Aktualisierung des Inhalts der Webseite und einen Zugang zum asynchronen HTTP-Client wird jQuery benutzt.
\section{Register-Client}
\label{sec:register_client}
Das folgende Kapitel beschäftigt sich mit der Implementierung des Register-Client auf Raspberry Pi Board mit angeschlossenen RFID-Leser. Dieser Teil der verteilte System lässt sich wie folgendes unterteilen. Zuerst wurde das Betriebssystem Raspbian auf Board zum Leben gebracht und dann die alle notwendigen für RFID-Leser Treiber installiert. Nach dem der RFID-Leser funktionieren angefangen und die Daten von RFID-Transponder abgelesen hat, wurde die nächste Herausforderungen gelöst: die Struktur die zu empfangenen Daten wurde verstanden, richtig bearbeitet, eine JSON-Datei erstellt und durch die HTTP-Protokoll dem Server geliefert. 

\subsection{Installation des Betriebssystem}
\label{sec:register_client:raspbian}
Der vorhandene für die Abschlussarbeit Raspberry Pi 3 Model B+ wurde nicht als Starter Kit mir übergeben, dann wurde es zusätzlich benötigt\cite[pp. 21-22]{gareth:raspi}: 
\begin{itemize}
	\item \textbf{USB-Netzteil} mit einer Nennleistung von 2,5 A (2,5 A) oder 12,5 Watt (12,5 W) und einem Micro-USB-Anschluss. 
	\item \textbf{microSD-Karte}, die als permanenter Speicher des Raspberry Pi dient; Alle von Benutzer erstellten Dateien und die installierte Software sowie das Betriebssystem selbst werden auf der microSD-Karte gespeichert.
	\item \textbf{USB-Tastatur und -Maus}, mit denen den Raspberry Pi gesteuert werden kann. Fast jede kabelgebundene oder kabellose Tastatur und Maus mit USB-Anschluss funktioniert mit dem Raspberry Pi.
	\item \textbf{Das HDMI-Kabel}, das Ton und Bilder vom Raspberry Pi auf Fernseher oder Monitor überträgt. Sie müssen nicht viel Geld für ein HDMI-Kabel ausgeben. 
\end{itemize}

Die Arbeit mit einem RaspberryPi setzt ein paar Anfangsinvestitionen voraus, die auch von den angestrebten Aufgaben und Projekten abhängen. Zuerst gäbe es die Möglichkeiten, dass der gekaufte Raspberry Pi Board bereits ein Betriebssystem darauf installiert hätte. Aber es war nicht der Fall von vorhandenen im PSE-Labor Board. Um ein Betriebssystem auf diesen Raspberry Pi zu bringen, muss eine SD-Karte mit einem Betriebssystem-Image "geflasht" werden. Dafür zunächst wurde die Distribution von der Website Raspbian.org herunterladen und die MicroSD-Karte in den Kartenleser eines vorhandenen im PSE-Labor PC eingelegt. Anschließend wurde mit dem Macintosh Disk Utility-Dienstprogramm das heruntergeladene und entpackte Betriebssystem für den RaspberryPi auf eine Speicherkarte geschrieben. Dann ist die Karte in Raspberry Pi einzulegen. Der Raspi ist damit betriebsbereit und muss für die zukünftigen Anwendungen noch konfiguriert werden. Wenn der Pi zum ersten Mal eingeschaltet wird, wird viel Text auf dem Bildschirm angezeigt. Diese werden als Startmeldungen bezeichnet. Wenn Raspbian zum ersten Mal gestartet wurde, kann es ein oder zwei Minuten dauern, um die Nutzung des freien Speicherplatzes auf der microSD-Karte optimal anzupassen. Beim nächsten Start geht es schneller. Schließlich ist kurz ein Fenster mit dem Raspberry Pi-Logo zu sehen, dann wird Terminal Fenster angezeigt, in dem es einloggt werden muss. Zum ersten Einloggen wird den Standardbenutzername "pi" und das Standardkennwort  "raspberry" verwendet. Um Register-Client vor sowohl Online-Bedrohungen als auch von Missbrauch im Labor zu schützen, wurde das Standardkennwort sofort geändert. Der nächste Schritt ist Raspbian bis zur Version "Raspbian mit dem Raspberry Pi Desktop" zu aktualisieren, damit die grafische Benutzeroberfläche und Chromium Browser zu Verfügung stehen können. Dies kann mit dem Terminalbefehl gemacht werden:
\begin{lstlisting}
sudo apt-get install lxde-core xserver-xorg xinit
\end{lstlisting}
Dann ist der Raspberry Pi erneut zu laden. Nachdem Raspberry Pi-Logo wieder angezeigt wurde, wäre der Raspbian-Desktop zu sehen. Somit gilt Betriebssystem als vollständig installiert und kann benutzt werden. Das war aber nicht der Fall mit dem vorhandenen Hardware, da es plötzlich eine Boot-Schleife vorkam, nachdem der Mikrocomputer eingeschaltet wurde und der Startvorgang nicht abgeschlossen werden konnte. Anstatt das zum Benutzung bereiteten Betriebssystem mit der grafische Benutzeroberfläche zu sehen,  wird eine Schleife erzeugt, in der die Startvorgang kontinuierlich und wiederholt ausgeführt wurde und somit eine Nutzung der Mikrocomputers unmöglich ist. Nach den mehreren Recherchen wurde es vermutet, dass es durch eine unzureichende Stromversorgung verursacht werden könnte. Es wurde aber zuerst nicht versucht, einen USB-Netzteil zu wechseln, da die anderen USB-Netzteil man durch PSE-Labor bestellen und eine Zeit abwarten muss. Jedoch wurde eine erzeugte Boot-Schleife mit einem anderen Terminalbefehl erfolgreich gelöst:
\begin{lstlisting}
sudo apt-get install --reinstall pcmanfm
\end{lstlisting}
Bei der Arbeit mit dem Mikrocomputer tritt jedoch später ein Problem mit der Stromversorgung auf. Der Fall kann im entsprechenden Kapitel \ref{sec:register_client:voltage_issue} nachgelesen werden.


\subsection{Installation der Treibers für RFID-Leser}
\label{sec:register_client:install_rfid}
Obwohl Raspbian mit einer Reihe von Software vorinstalliert ist, wird es aber zusätzlich benötigt, die Treiber für RFID-Leser zu installieren. 



\subsection{Spannungsproblem und Lösung}
\label{sec:register_client:voltage_issue}
Das offizielle Raspberry Pi-Netzteil ist die empfohlene Wahl, da es den schnell wechselnden Strombedarf des Raspberry Pi bewältigen kann.  KEYBOARD bwohl einige Tastaturen im Gaming-Stil mit bunten Lichtern möglicherweise zu viel Strom verbrauchen, um zuverlässig verwendet zu werden.

\section{Server}
\label{sec:server}


\section{Display-Client}
\label{sec:display_client}
